
% This LaTeX was auto-generated from an M-file by MATLAB.
% To make changes, update the M-file and republish this document.

    
    
\section*{Hobby's splines, example}

This file demonstrates some examples of drawing Bezier curves in Matlab using Hobby's algorithm for choosing control points based on curvature and tension.



\subsection*{Simple examples}

\begin{lstlisting}[style=matlab]
% Create a new figure:
hfig = figure(1);
set(hfig,'color',[1 1 1],'name','Spline example');
clf;

% For clarity, define points separately:
points = {...
 [0 0],[1 1],[2,0],...
};
hobbysplines(points,'debug',true,'cycle',true);

% Experiment with `tension'. Note the default creates roughly circular
% plots.
points = {[1 0],[1 1],[0 1],[0 0]};
hobbysplines(points,'debug',true,'cycle',true,'offset',[3 0]);
hobbysplines(points,'debug',true,'tension',1,'cycle',true,'offset',[3 0],'linestyle','.');
hobbysplines(points,'debug',true,'tension',2,'cycle',true,'offset',[3 0],'linestyle','--');

axis equal
axis off
\end{lstlisting}

\includegraphics [width=4in]{splines_example_01.eps}


\subsection*{Example with more points}

\begin{lstlisting}[style=matlab]
hfig = figure(2); clf;
set(hfig,'color',[1 1 1],'name','Spline example');

points = {{[0 0] '' 1 1},[0.7 0.8],[0.8 2],[1 4],[0 5],[-1 3.5],[-0.8 2],[-0.8 1]};
hobbysplines(points,'debug',true,'tension',2,'cycle',true);

axis equal
axis off
\end{lstlisting}

\includegraphics [width=4in]{splines_example_02.eps}
